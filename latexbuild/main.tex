\documentclass[a4paper,landscape,twocolumn]{article} % 横向页面, 双栏
\usepackage[utf8]{inputenc}

\title{codebook}
\author{tieway59}
\date{September 2020}

% 页码位置调整最简单方案
% https://tex.stackexchange.com/questions/443344/page-numbering-left-and-right
\usepackage{titleps}
\renewpagestyle{plain}{
    \sethead{}{}{}
    \setfoot[github.com/TieWay59/tmplz][ - \thepage \ -  ][\today]{github.com/TieWay59/tmplz}{ - \thepage \ - }{\today}
    % 分别是奇偶页的定位
}
\pagestyle{plain}


% 页边距设置
\usepackage{geometry}
\geometry{a4paper, left=0.7cm, right=0.7cm, top=1.3cm, bottom=8mm, ,footskip=5mm}


% 字体设置 https://github.com/source-foundry/Hack/issues/69#issuecomment-138621789
\usepackage{fontspec}
% \setmonofont{Liberation Mono}
\setmonofont[
    Path=./fonts/ ,
    Extension = .ttf, % or .otf
    UprightFont = *-regular,
    BoldFont = *-bold,
    ItalicFont = *-italic,
    BoldItalicFont = *-bolditalic,
    % Ligatures = %TeX, 这个问题无法解决,就是xelatex编译的时候,<=之类的符号会连字,但是应该页不影响阅读。使用lua就没毛病。
]{sarasa-term-sc}

\setmainfont[
    Path=./fonts/ ,
    Extension = .ttf, % or .otf
    UprightFont = *-regular,
    BoldFont = *-bold,
    ItalicFont = *-italic,
    BoldItalicFont = *-bolditalic,
    % Ligatures = %TeX, 这个问题无法解决,就是xelatex编译的时候,<=之类的符号会连字,但是应该页不影响阅读。使用lua就没毛病。
]{sarasa-term-sc}
% \setmonofont{sarasa-term-sc-regular.ttf}

\usepackage{minted}
\setminted{
    escapeinside = false,
    mathescape,             % 注释支持公式
    linenos = true,         % 行号
    breaklines = true,      % 自动折行
    encoding = utf8,
    fontsize = \small,      % 内容字体大小
    numbersep = 3pt,        % 行号数字偏移
    xleftmargin = 8pt,      % 代码块左侧外部偏移
    frame = single,         % 代码块线框样式
    framerule = 0.4pt,      % 线框宽度
    framesep = 1mm          % 内容与线框偏移
}

% \usemintedstyle{xcode}      % 颜色样式(这个报红较少)
\usemintedstyle{bw}         % 黑白
% \usemintedstyle{vs}         % vs

% 设置行号字体
% \usepackage{./packages/fancyvrb}
% \renewcommand{\theFancyVerbLine}{\monofont{\scriptsize\arabic{FancyVerbLine}}}  % oneleaf会对这个指令爆出很多错误,但是这个是有效的。 看着烦可以关掉,就是行号有点小。

% 进一步避免报红 https://tex.stackexchange.com/questions/343494/minted-red-box-around-greek-characters
\usepackage{etoolbox,xpatch}
\makeatletter
\AtBeginEnvironment{minted}{\dontdofcolorbox}
\def\dontdofcolorbox{\renewcommand\fcolorbox[4][]{##4}}
\xpatchcmd{\inputminted}{\minted@fvset}{\minted@fvset\dontdofcolorbox}{}{}
\xpatchcmd{\mintinline}{\minted@fvset}{\minted@fvset\dontdofcolorbox}{}{} % see https://tex.stackexchange.com/a/401250/
\makeatother


\begin{document}

% \maketitle 加了第一页有点问题,会浪费两张纸,所以我就干脆不要了。

\begingroup
\let\onecolumn\twocolumn
\tableofcontents
\endgroup

\newpage

\section{pb-ds}
\subsection{common-sample.cpp}
\inputminted{c++}{./codes/000}
\section{stl}
\subsection{unordered-map}
\subsubsection{1.txt}
\inputminted{text}{./codes/001}
\subsubsection{lgoj3370.cpp}
\inputminted{c++}{./codes/002}
\section{图论}
\subsection{k短路}
\subsubsection{lgoj2483.cpp}
\inputminted{c++}{./codes/003}
\subsubsection{lgoj4467.cpp}
\inputminted{c++}{./codes/004}
\subsection{二分图最大匹配}
\subsubsection{hungaryO(n(e+m)).cpp}
\inputminted{c++}{./codes/005}
\subsection{二分图最大权匹配}
\subsubsection{kmO(n^3).cpp}
\inputminted{c++}{./codes/006}
\subsection{强联通缩点}
\subsubsection{lgoj3387.cpp}
\inputminted{c++}{./codes/007}
\subsection{拓扑排序}
\subsubsection{lgoj1954.cpp}
\inputminted{c++}{./codes/008}
\subsection{最小生成树}
\subsubsection{boruvka}
\subsubsection{cf888g.cpp}
\inputminted{c++}{./codes/009}
\subsection{最短路}
\subsubsection{dijkstra}
\subsubsection{lgoj4779.cpp}
\inputminted{c++}{./codes/010}
\subsubsection{Johnson}
\subsubsection{lgoj5905.cpp}
\inputminted{c++}{./codes/011}
\section{字符串}
\subsection{AC自动机}
\subsubsection{lgoj5357.cpp}
\inputminted{c++}{./codes/012}
\subsection{bordertree}
\subsubsection{lgoj5829.cpp}
\inputminted{c++}{./codes/013}
\subsection{exkmp}
\subsubsection{exkmp.cpp}
\inputminted{c++}{./codes/014}
\subsection{kmp}
\subsubsection{lgoj3375.cpp}
\inputminted{c++}{./codes/015}
\subsection{lyndon}
\subsubsection{hdoj6761.cpp}
\inputminted{c++}{./codes/016}
\subsubsection{lgoj1368.cpp}
\inputminted{c++}{./codes/017}
\subsubsection{lgoj6114.cpp}
\inputminted{c++}{./codes/018}
\subsubsection{uva719.cpp}
\inputminted{c++}{./codes/019}
\subsection{manacher}
\subsubsection{lgoj3805.cpp}
\inputminted{c++}{./codes/020}
\subsection{trie}
\subsubsection{lgoj3370.cpp}
\inputminted{c++}{./codes/021}
\subsection{后缀数组nlogn}
\subsubsection{suffixarray.cpp}
\inputminted{c++}{./codes/022}
\subsection{后缀自动机}
\subsubsection{lgoj3804.cpp}
\inputminted{c++}{./codes/023}
\subsubsection{lgoj3804.md}
\inputminted{text}{./codes/024}
\subsubsection{lgoj3975.cpp}
\inputminted{c++}{./codes/025}
\subsubsection{lgoj4070.cpp}
\inputminted{c++}{./codes/026}
\subsubsection{sam(simple).md}
\inputminted{text}{./codes/027}
\subsection{回文自动机}
\subsubsection{lgoj3649.cpp}
\inputminted{c++}{./codes/028}
\subsection{子序列}
\subsubsection{lgoj5826.cpp}
\inputminted{c++}{./codes/029}
\subsection{广义后缀自动机}
\subsubsection{lgoj3181.cpp}
\inputminted{c++}{./codes/030}
\subsubsection{lgoj3346.cpp}
\inputminted{c++}{./codes/031}
\subsubsection{lgoj4022.cpp}
\inputminted{c++}{./codes/032}
\subsubsection{lgoj6139.cpp}
\inputminted{c++}{./codes/033}
\subsubsection{OvO.cpp}
\inputminted{c++}{./codes/034}
\subsection{最小表示}
\subsubsection{lgoj1368.cpp}
\inputminted{c++}{./codes/035}
\section{数据结构}
\subsection{odt}
\subsubsection{CF896C.cpp}
\inputminted{c++}{./codes/036}
\subsubsection{CF915E.cpp}
\inputminted{c++}{./codes/037}
\subsubsection{lgoj4979.cpp}
\inputminted{c++}{./codes/038}
\subsection{左偏树}
\subsubsection{lgoj3377.cpp}
\inputminted{c++}{./codes/039}
\subsection{平衡树}
\subsubsection{fhqtreap}
\subsubsection{3369.cpp}
\inputminted{c++}{./codes/040}
\subsubsection{3391.cpp}
\inputminted{c++}{./codes/041}
\subsubsection{3960.cpp}
\inputminted{c++}{./codes/042}
\subsubsection{6136.cpp}
\inputminted{c++}{./codes/043}
\subsection{树套树}
\subsubsection{lgoj2617.cpp}
\inputminted{c++}{./codes/044}
\subsection{树状数组}
\subsubsection{区间修改单点查询}
\subsubsection{lgoj3368.cpp}
\inputminted{c++}{./codes/045}
\subsubsection{单点修改区间查询}
\subsubsection{lgoj3374.cpp}
\inputminted{c++}{./codes/046}
\subsection{笛卡尔树}
\subsubsection{5854.cpp}
\inputminted{c++}{./codes/047}
\subsection{线段树}
\subsubsection{区间加minmax历史最值}
\subsubsection{uoj170.cpp}
\inputminted{c++}{./codes/048}
\subsubsection{区间加乘求和}
\subsubsection{lgoj3373.cpp}
\inputminted{c++}{./codes/049}
\subsubsection{区间加取min求和历史最值}
\subsubsection{lgoj6242.cpp}
\inputminted{c++}{./codes/050}
\subsubsection{区间加求和}
\subsubsection{lgoj3372.cpp}
\inputminted{c++}{./codes/051}
\subsubsection{可持久化线段树合并}
\subsubsection{线段树合并}
\subsubsection{lgoj4556.cpp}
\inputminted{c++}{./codes/052}
\section{数论}
\subsection{区间lcm}
\subsubsection{lgoj5655.cpp}
\inputminted{c++}{./codes/053}
\subsection{多项式}
\subsubsection{fft&ntt}
\subsubsection{3803.cpp}
\inputminted{c++}{./codes/054}
\subsubsection{fwt}
\subsubsection{fwt.cpp}
\inputminted{c++}{./codes/055}
\section{模拟退火}
\subsection{lgoj4035(unfinished).cpp}
\inputminted{c++}{./codes/056}
\subsection{note.md}
\inputminted{text}{./codes/057}
\subsection{note.txt}
\inputminted{text}{./codes/058}
\subsection{UVA10228.cpp}
\inputminted{c++}{./codes/059}

\end{document}
