\documentclass[a4paper,landscape,twocolumn]{article} % 横向页面, 双栏
\usepackage[utf8]{inputenc}

\title{codebook}
\author{tieway59}
\date{September 2020}

% 页码位置调整最简单方案
% https://tex.stackexchange.com/questions/443344/page-numbering-left-and-right
\usepackage{titleps}
\renewpagestyle{plain}{
    \sethead{}{}{}
    \setfoot[github.com/TieWay59/tmplz][ - \thepage \ -  ][\today]{github.com/TieWay59/tmplz}{ - \thepage \ - }{\today}
    % 分别是奇偶页的定位
}
\pagestyle{plain}


% 页边距设置
\usepackage{geometry}
\geometry{a4paper, left=0.7cm, right=0.7cm, top=1.3cm, bottom=8mm, ,footskip=5mm}


% 字体设置 https://github.com/source-foundry/Hack/issues/69#issuecomment-138621789
\usepackage{fontspec}
% \setmonofont{Liberation Mono}
\setmonofont[
    Path=./fonts/ ,
    Extension = .ttf, % or .otf
    UprightFont = *-regular,
    BoldFont = *-bold,
    ItalicFont = *-italic,
    BoldItalicFont = *-bolditalic,
    % Ligatures = %TeX, 这个问题无法解决,就是xelatex编译的时候,<=之类的符号会连字,但是应该页不影响阅读。使用lua就没毛病。
]{sarasa-term-sc}

\setmainfont[
    Path=./fonts/ ,
    Extension = .ttf, % or .otf
    UprightFont = *-regular,
    BoldFont = *-bold,
    ItalicFont = *-italic,
    BoldItalicFont = *-bolditalic,
    % Ligatures = %TeX, 这个问题无法解决,就是xelatex编译的时候,<=之类的符号会连字,但是应该页不影响阅读。使用lua就没毛病。
]{sarasa-term-sc}
% \setmonofont{sarasa-term-sc-regular.ttf}

\usepackage{minted}
\setminted{
    escapeinside = false,
    mathescape,             % 注释支持公式
    linenos = true,         % 行号
    breaklines = true,      % 自动折行
    encoding = utf8,
    fontsize = \small,      % 内容字体大小
    numbersep = 3pt,        % 行号数字偏移
    xleftmargin = 8pt,      % 代码块左侧外部偏移
    frame = single,         % 代码块线框样式
    framerule = 0.4pt,      % 线框宽度
    framesep = 1mm          % 内容与线框偏移
}

% \usemintedstyle{xcode}      % 颜色样式(这个报红较少)
\usemintedstyle{bw}         % 黑白
% \usemintedstyle{vs}         % vs

% 设置行号字体
% \usepackage{./packages/fancyvrb}
% \renewcommand{\theFancyVerbLine}{\monofont{\scriptsize\arabic{FancyVerbLine}}}  % oneleaf会对这个指令爆出很多错误,但是这个是有效的。 看着烦可以关掉,就是行号有点小。

% 进一步避免报红 https://tex.stackexchange.com/questions/343494/minted-red-box-around-greek-characters
\usepackage{etoolbox,xpatch}
\makeatletter
\AtBeginEnvironment{minted}{\dontdofcolorbox}
\def\dontdofcolorbox{\renewcommand\fcolorbox[4][]{##4}}
\xpatchcmd{\inputminted}{\minted@fvset}{\minted@fvset\dontdofcolorbox}{}{}
\xpatchcmd{\mintinline}{\minted@fvset}{\minted@fvset\dontdofcolorbox}{}{} % see https://tex.stackexchange.com/a/401250/
\makeatother


\begin{document}

% \maketitle 加了第一页有点问题,会浪费两张纸,所以我就干脆不要了。

\begingroup
\let\onecolumn\twocolumn
\tableofcontents
\endgroup

\newpage

\section{字符串}
\subsection{Aho–Corasick-AC自动机-多模式匹配.cpp}
\inputminted{c++}{./codes/000}
\subsection{Aho–Corasick-AC自动机-统计次数-拓扑序优化.cpp}
\inputminted{c++}{./codes/001}
\subsection{Levenshtein-Distance-编辑距离.py}
\inputminted{python}{./codes/002}
\subsection{manacher-单数组马拉车.cpp}
\inputminted{c++}{./codes/003}
\subsection{manacher-双数组马拉车.cpp}
\inputminted{c++}{./codes/004}
\subsection{PAM-回文串后缀-1.cpp}
\inputminted{c++}{./codes/005}
\subsection{PAM-回文串哈希-1.cpp}
\inputminted{c++}{./codes/006}
\subsection{PAM-回文串统计-1.cpp}
\inputminted{c++}{./codes/007}
\subsection{PAM-回文串统计-2.cpp}
\inputminted{c++}{./codes/008}
\subsection{PAM-回文串统计-3.cpp}
\inputminted{c++}{./codes/009}
\subsection{SAM-子串匹配-1.cpp}
\inputminted{c++}{./codes/010}
\subsection{SAM-子串统计-1.cpp}
\inputminted{c++}{./codes/011}
\subsection{SAM-子串统计-2.cpp}
\inputminted{c++}{./codes/012}
\section{数据结构}
\subsection{zhuxishu-SegKth.cpp}
\inputminted{c++}{./codes/013}
\subsection{ZTC's-Splay.cpp}
\inputminted{c++}{./codes/014}
\subsection{主席树-区间统计小于等于k.cpp}
\inputminted{c++}{./codes/015}
\section{日期}
\subsection{CommonSense-常识.cpp}
\inputminted{c++}{./codes/016}
\subsection{KimLarsen-日期算星期.cpp}
\inputminted{c++}{./codes/017}
\section{杂项}
\subsection{coutf.cpp}
\inputminted{c++}{./codes/018}
\subsection{debug-from-tourist.cpp}
\inputminted{c++}{./codes/019}
\subsection{fast-IO-int.cpp}
\inputminted{c++}{./codes/020}
\subsection{fast-IO-ZTC.cpp.ignore}
\subsection{fast-IO-快速版(可敲).cpp.ignore}
\subsection{fast-IO-极致版.ignore}
\subsection{fastpow-快速幂.cpp}
\inputminted{c++}{./codes/024}
\subsection{Misc-杂技---随机数.cpp}
\inputminted{c++}{./codes/025}
\subsection{string-read-speed.cpp.ignore}
\subsection{timetest.cpp}
\inputminted{c++}{./codes/027}
\subsection{unordered-map-自写哈希.cpp}
\inputminted{c++}{./codes/028}
\subsection{单调队列-定长区间最值.cpp}
\inputminted{c++}{./codes/029}

\end{document}
